\documentclass[utf8,bachelor]{gradu3}
% If you are writing a Bachelor's Thesis, use the following instead:
%\documentclass[utf8,bachelor,english]{gradu3}

\usepackage{graphicx} % for including pictures

\usepackage{amsmath} % useful for math (optional)

\usepackage{booktabs} % good for beautiful tables

% NOTE: This must be the last \usepackage in the whole document!
\usepackage[bookmarksopen,bookmarksnumbered,linktocpage]{hyperref}

\addbibresource{references.bib} % The file name of your bibliography database

\begin{document}

\title{Lohkoketjut internetissä}
\translatedtitle{Blockchains in the internet}
\studyline{Tietotekniikka}
\avainsanat{%
  lohkoketjut,
  dApp,
  hajautettu internet,
  web3
  }
\keywords{
    blockchain,
    decentralized application,
    web3.0,
    NFT,
    smart contract
}
\tiivistelma{%
Tässä paperissä käsitellään lohkoketjuja internetissä sekä lohkoketjujen hyötyjä, ongelmia ja haittoja.
}
\abstract{%
This paper contains information of blockchain
}

\author{Niko Sihvo}
\contactinformation{\texttt{niko.m.sihvo@student.jyu.fi}}
% use a separate \author command for each author, if there is more than one
\supervisor{Unsupervised work}
% use a separate \supervisor command for each supervisor, if there
% is more than one

 % you don't need this line in a thesis
% \type{Template and manual for a thesis document class}

\maketitle


\begin{thetermlist}
\item[Blockchain] lohkoketju
\item[\emph{Miner}] louhija
\item[DApp] decentralized application
\item[NFT] non-fungible token
\item{node} solmu
\end{thetermlist}

\mainmatter

\chapter{Johdanto}




\chapter{Lohkoketju}
\section{Historia}

Lohkoketjuteknologian perusajatus on lähtöisin 80- ja 90-luvun vaihteesta. Leslie Lambort kehitti Paxos protokollan vuonna 1989, mutta kyseinen \emph{The Part-Time Parliament} \parencite{lamport2019part} tutkielma julkaistiin ACM Transactions on Computers Systems -julkaisussa vuonna 1998. Tutkielma kuvailee konsensusmallin miten verkko tietokoneita voivat tulla yhteisymmärrykseen, missä tietokone tai verkko itsessään on epäluotettava.

Vuonna 1991 allekirjoitettua informaatioketjua alettiin käyttämään elektronisena kirjanpitona dokumenttien digitaaliseen allekirjoittamiseen.
Tämä mahdollisti tavan nähdä ettei yhtäkään allekirjoitettua dokumenttia oltu muokattu kokoelmassa.

Nämä kaksi konseptia sovellettiin yhteen luoden ensimmäisen lohkoketjusovelluksen Bitcoinin. Bitcoin esiteltiin Satoshi Nakamoton artikkelissa \emph{Bitcoin: A Peer to Peer Electronic Cash System} \parencite{nakamoto2008bitcoin} vuonna 2008.
Nakamoton artikkeli on monien modernien kryptovaluuttojen perusta.
Bitcoin on ensimmäisiä lohkoketjusovelluksia siten myös ensimmäisiä hajautettuja sovelluksia.

Lohkoketjujen käyttö mahdollisti Bitcoinin käytön hajautetulla tavalla, missä yksittäinen käyttäjä ei ohjaa elektronista rahaa eikä järjestelmässä ole yksittäistä epäonnistumispistettä.
Lohkoketjun ensisijainen hyöty on mahdollistaa suorat transaktiot käyttäjien välillä ilman kolmansia osapuolia.
Lohkoketju mahdollistaa määritellyn tavan palkita niitä käyttäjiä, jotka onnistuvat julkaisemaan uusia lohkoja ja ylläpitää tilikirjan kopiota. Tälläisiä käyttäjia kutsutaan louhijoiksi.
Louhijoiden automatisoitu maksu mahdollistaa systeemin hajautetun hallinnon ilman tarvettta järjestäytyä.
Tämä itsevalvottu mekanismi, joka käyttää lohkoketjua ja konsensusperustaista ylläpitoa, vakuuttaa että vain validit transaktiot ja lohkot lisätään lohkoketjuun.

Lohkoketju mahdollistaa käyttäjien salauksen. 
Käyttäjät ovat anonyymeja, mutta heidän käyttäjätunnisteet eivät ole. Julkisessa lohkoketjussa kaikki transaktiot ovat julkisia. Koska lohkoketjutsovellukset ovat yleensä salattuja, on olennaista olla mekanismit luottamuksen luotiin ympäristössä, jossa käyttäjiä ei voi tunnistaa.
Ilman luotettavia välikäsiä, lohkoketjuverkot mahdollistaa luottamuksen luonnin lohkoketjuteknologian neljän avainominaisuuden avulla.
\parencite{yaga2019blockchain}

Hajauttaminen, missä jokainen lohkoketjuverkossa tehty transaktio on vain kahden noden välillä tehty siirto ilman välikäsiä.

Persistency, missä jokainen transaktio täytyy validoida luotetuilla louhijoilla. Persistency vakuuttaa että nodeihin tallennetut tilikirjat pysyy absoluuttisina ja muuttumattomina, eikä niintä pysty poistamaan.

Anonymiteetti, missä jokaisella louhijalla on generoitu osoite uniikkina indentiteettinä. Vaikka kaikki lohkoketjut ei ole anonyymeja kokonaa j a jotkut harjoittaa pseudo-anonymiteettiä, kuten Bitcoin ja Ethereum. 

Tarkastettavuus




\parencite{zarrin2021blockchain}






\chapter{Yhteenveto}


\printbibliography

\appendix


\end{document}
